%%%%%%%%%%%%%%%%%%%%%%%%%%%%%%%%%%%%%%%%%
% Short Sectioned Assignment
% LaTeX Template
% Version 1.0 (5/5/12)
%
% This template has been downloaded from:
% http://www.LaTeXTemplates.com
%
% Original author:
% Frits Wenneker (http://www.howtotex.com)
%
% License:
% CC BY-NC-SA 3.0 (http://creativecommons.org/licenses/by-nc-sa/3.0/)
%
%%%%%%%%%%%%%%%%%%%%%%%%%%%%%%%%%%%%%%%%%

%----------------------------------------------------------------------------------------
%	PACKAGES AND OTHER DOCUMENT CONFIGURATIONS
%----------------------------------------------------------------------------------------

\documentclass[titlepage, paper=a4, fontsize=11pt]{scrartcl} % A4 paper and 11pt font size

\usepackage[T1]{fontenc} % Use 8-bit encoding that has 256 glyphs
\usepackage{fourier} % Use the Adobe Utopia font for the document - comment this line to return to the LaTeX default
\usepackage[english]{babel} % English language/hyphenation
\usepackage{amsmath,amsfonts,amsthm} % Math packages
\usepackage{listings}

\usepackage{lipsum} % Used for inserting dummy 'Lorem ipsum' text into the template


\usepackage{sectsty} % Allows customizing section commands
\allsectionsfont{\centering \normalfont\scshape} % Make all sections centered, the default font and small caps

\usepackage{fancyhdr} % Custom headers and footers
\pagestyle{fancyplain} % Makes all pages in the document conform to the custom headers and footers
\fancyhead{} % No page header - if you want one, create it in the same way as the footers below
\fancyfoot[L]{} % Empty left footer
\fancyfoot[C]{} % Empty center footer
\fancyfoot[R]{\thepage} % Page numbering for right footer
\renewcommand{\headrulewidth}{0pt} % Remove header underlines
\renewcommand{\footrulewidth}{0pt} % Remove footer underlines
\setlength{\headheight}{13.6pt} % Customize the height of the header

\numberwithin{equation}{section} % Number equations within sections (i.e. 1.1, 1.2, 2.1, 2.2 instead of 1, 2, 3, 4)
\numberwithin{figure}{section} % Number figures within sections (i.e. 1.1, 1.2, 2.1, 2.2 instead of 1, 2, 3, 4)
\numberwithin{table}{section} % Number tables within sections (i.e. 1.1, 1.2, 2.1, 2.2 instead of 1, 2, 3, 4)

\setlength\parindent{0pt} % Removes all indentation from paragraphs - comment this line for an assignment with lots of text

%----------------------------------------------------------------------------------------
%	TITLE SECTION
%----------------------------------------------------------------------------------------

\newcommand{\horrule}[1]{\rule{\linewidth}{#1}} % Create horizontal rule command with 1 argument of height

\title{	
\normalfont \normalsize 
\textsc{University of Virginia} \\ [25pt] % Your university, school and/or department name(s)
\horrule{0.5pt} \\[0.4cm] % Thin top horizontal rule
\huge CS 6161 Algorithms \\
\huge Problem Set 1 \\ % The assignment title
\horrule{2pt} \\[0.5cm] % Thick bottom horizontal rule
}

\author{Shawn (Shuoshuo) Chen\\sc7cq@virginia.edu\\Group partner: Rolph Recto\\ rjr7je@virginia.edu} % Your name

\date{\normalsize\today} % Today's date or a custom date

\begin{document}

\maketitle % Print the title

%----------------------------------------------------------------------------------------
%	PROBLEM 1
%----------------------------------------------------------------------------------------

\section*{Problem 1}
\textbf{(a)}.
Assuming the value of ${x_1,x_2, ... ,x_n}$ is pick at random independently. Then the probability of getting true
and false is equal, where $p_{true} = p_{false} = 0.5$. For a clause $C_i$, the probability of evaluating true is
$E[p_{true}] = 1-2^{-d}$. And there are $m$ clauses in $\phi$, so the number of satified clauses is
$E[N_{SAT}] = (1-2^{-d}) * m$, which indicates there is an assignment satisfying at least $(1-2^{-d})m$ clauses.
\\

\textbf{(b)}. By definition, $|B(i)| = Y = \sum_{j \in [n]} Y_j$, thus $Var[|B(i)|] = Var[Y]$. And there is \\
\begin{align*} 
\begin{split}
Var[Y] &= \mathbb{E}[Y^2] - (\mathbb{E}[Y])^2 \\
&= \mathbb{E}[(\sum_{j \in [n]} Y_j)^2] - (\mathbb{E}[\sum_{j \in [n]} Y_j])^2 \\
&= \mathbb{E}[(\sum_{j \in [n]} Y_j)^2] - (\sum_{j \in [n]} \mathbb{E}[Y_j])^2 \\
&= \mathbb{E}[(\sum_{j \in [n]} Y_j)^2] - n \times \frac{1}{n} \\
&= \mathbb{E}[\sum_{j \in [n]}\sum_{i \in [n]} Y_iY_j] - 1 \\
&= \mathbb{E}[\sum_{j \in [n]} Y_i^2 + \sum_{j \in [n]}\sum_{i \in [n], i \neq j} Y_iY_j] - 1 \\
&= \sum_{j \in [n]}\mathbb{E}[Y_i^2] + \sum_{j \in [n]}\sum_{i \in [n], i \neq j}\mathbb{E}[Y_iY_j] -1 \\
&= \sum_{j \in [n]}\mathbb{E}[Y_i] + \sum_{j \in [n]}\sum_{i \in [n], i \neq j}\mathbb{E}[Y_i]\mathbb{E}[Y_j] -1 \\
&= n \times \frac{1}{n} + \frac{n(n-1)}{2} \times 2 \times \frac{1}{n^2} -1 \\
&= \frac{n-1}{n} \\
&= 1 - \frac{1}{n}
\end{split}					
\end{align*}
As we can see, with $n \to \infty$, $Var[Y] \to 1$. And for any $n$, $Var[Y] \leqslant 1$. \\
That is, $Var[|B(i)|] = O(1)$.
\\

\textbf{(c)}. The Chebyshev's inequality gives
\begin{align*} 
\begin{split}
Pr[|X-\mu| \geqslant k\delta] \leqslant \frac{1}{k^2}
\end{split}					
\end{align*}
We can replace $X = |B(j)|$, $\mu = 1$ and $\delta = \sqrt{\frac{2n-1}{n}}$.
\begin{align*} 
\begin{split}
&Pr[|\ |B(j)|-1\ | \geqslant k\sqrt{\frac{2n-1}{n}}] \leqslant \frac{1}{k^2} \\
&Pr[|B(j)| \geqslant 1+k\sqrt{\frac{2n-1}{n}}] \leqslant \frac{1}{k^2} \\
\end{split}					
\end{align*}
If we make $k^2 = 100n$, solve for k, we get $k = 10\sqrt{n}$.
Replace $k = 10\sqrt{n}$:
\begin{align*} 
\begin{split}
Pr[|B(j)| \geqslant 1+10\sqrt{2n-1}] \leqslant \frac{1}{100n}
\end{split}					
\end{align*}
And it is easy to see that:
\begin{align*} 
\begin{split}
Pr[|B(j)| \textgreater c\sqrt{n}] \leqslant \frac{1}{100n} \ \bigg|_{c\sqrt{n} \ \textgreater \ 1+10\sqrt{2n-1}}
\end{split}					
\end{align*}
Solve for c in $c\sqrt{n} \ \textgreater \ 1+10\sqrt{2n-1}$, we get $c \textgreater \frac{1}{\sqrt{n}} + 10\sqrt{\frac{2n-1}{n}}$. \\
Note that $c > 10\sqrt{2} \ \bigg|_{n \to \infty}$. \\
To sum up, for every fixed $j \in [k=n]$, there is
\begin{align*} 
\begin{split}
Pr[|B(j)| \textgreater c\sqrt{n}] \leqslant \frac{1}{100n} \ \bigg|_{c \ \textgreater \ 10\sqrt{2}}
\end{split}					
\end{align*}
Applying the union bound over this upper bound for $|B(1)|, |B(2)|, ..., |B(n)|$, we get
\begin{align*} 
\begin{split}
Pr[\forall j \in [n]: |B(j)| \textgreater c\sqrt{n}] \leqslant \frac{1}{100} \ \bigg|_{c \ \textgreater \ 10\sqrt{2}}
\end{split}					
\end{align*}
Flip it over, we have
\begin{align*} 
\begin{split}
Pr[\forall j \in [n]: |B(j)| \leqslant c\sqrt{n}] \geqslant \frac{99}{100} \ \bigg|_{c \ \textgreater \ 10\sqrt{2}}
\end{split}					
\end{align*}
\\


%----------------------------------------------------------------------------------------
%	PROBLEM 4
%----------------------------------------------------------------------------------------

\section*{Problem 4}
\textbf{18.16 (a).}
According to $D = P(H) * E[D | H] + P(T) * E[D | T] = H * D + T * (H * D + T)$ where $D$ is the state of both the start point and end point of H and TH in a tree diagram, there is:
\begin{align*} 
\begin{split}
E[N_{TT}] &= 0.5 * E[N_{TT} | H] + 0.5 * E[N_{TT} | T] \\
&= 0.5 * (1 + N_{TT}) + 0.5 * [1 + E[N_{TT} | 2nd]] \\
&= 0.5 * (1 + N_{TT}) + 0.5 * [1 + 0.5 * 1 + 0.5 * (1 + N_{TT})]
\end{split}					
\end{align*}
Solve for $N_{TT}$ and we get $N_{TT} = 6$.
Note that if we get a head (H), it basically adds one toss to the expectation. On the other hand, if we get a tail (T) followed by a head (H), it adds one toss to $E[N_{T*}]$, where $E[N_{T*}]$ is a combination of TT and TH. If we reach TT, then that's the end, just add another toss. Otherewise we need to start over again, so just plug the same structure $1 + N_{TT}$ in.
\\
\textbf{18.16 (b).}
Similarly, there is $C = H * C + T * B$ where $B = H + T * B$ for the tree diagram. We can derive:
\begin{align*} 
\begin{split}
E[N_{TH}] &= 0.5 * ( 1 + E[N_{TH} | H] ) + 0.5 * E[N_{B} | T] \\
\end{split}					
\end{align*}
And for $E_{B}$, there is:
\begin{align*} 
\begin{split}
E[N_{B}] &= 0.5 * 1 + 0.5 * ( 1 + E[N_{B}] ) \\
\end{split}					
\end{align*}
Solve for $E_{B}$ we get $E_{B} = 2$, and plug it into the $E_{TH}$ equation we get $E_{TH} = 4$.
\\
\textbf{18.16 (c).}
Since TT and TH both have the same first toss T, and by probability, the chance of getting T or H in the second toss is equally half. So $P(TT) = P(TH) = 0.5$. Then we can get the expected profit:
\begin{align*} 
\begin{split}
E[Profit] &= 0.5 * 6 + 0.5 * (-5) \\
&= 0.5 \quad dollars
\end{split}					
\end{align*}



\end{document}
