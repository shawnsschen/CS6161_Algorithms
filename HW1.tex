%%%%%%%%%%%%%%%%%%%%%%%%%%%%%%%%%%%%%%%%%
% Short Sectioned Assignment
% LaTeX Template
% Version 1.0 (5/5/12)
%
% This template has been downloaded from:
% http://www.LaTeXTemplates.com
%
% Original author:
% Frits Wenneker (http://www.howtotex.com)
%
% License:
% CC BY-NC-SA 3.0 (http://creativecommons.org/licenses/by-nc-sa/3.0/)
%
%%%%%%%%%%%%%%%%%%%%%%%%%%%%%%%%%%%%%%%%%

%----------------------------------------------------------------------------------------
%	PACKAGES AND OTHER DOCUMENT CONFIGURATIONS
%----------------------------------------------------------------------------------------

\documentclass[titlepage, paper=a4, fontsize=11pt]{scrartcl} % A4 paper and 11pt font size

\usepackage[T1]{fontenc} % Use 8-bit encoding that has 256 glyphs
\usepackage{fourier} % Use the Adobe Utopia font for the document - comment this line to return to the LaTeX default
\usepackage[english]{babel} % English language/hyphenation
\usepackage{amsmath,amsfonts,amsthm} % Math packages
\usepackage{listings}

\usepackage{lipsum} % Used for inserting dummy 'Lorem ipsum' text into the template


\usepackage{sectsty} % Allows customizing section commands
\allsectionsfont{\centering \normalfont\scshape} % Make all sections centered, the default font and small caps

\usepackage{fancyhdr} % Custom headers and footers
\pagestyle{fancyplain} % Makes all pages in the document conform to the custom headers and footers
\fancyhead{} % No page header - if you want one, create it in the same way as the footers below
\fancyfoot[L]{} % Empty left footer
\fancyfoot[C]{} % Empty center footer
\fancyfoot[R]{\thepage} % Page numbering for right footer
\renewcommand{\headrulewidth}{0pt} % Remove header underlines
\renewcommand{\footrulewidth}{0pt} % Remove footer underlines
\setlength{\headheight}{13.6pt} % Customize the height of the header

\numberwithin{equation}{section} % Number equations within sections (i.e. 1.1, 1.2, 2.1, 2.2 instead of 1, 2, 3, 4)
\numberwithin{figure}{section} % Number figures within sections (i.e. 1.1, 1.2, 2.1, 2.2 instead of 1, 2, 3, 4)
\numberwithin{table}{section} % Number tables within sections (i.e. 1.1, 1.2, 2.1, 2.2 instead of 1, 2, 3, 4)

\setlength\parindent{0pt} % Removes all indentation from paragraphs - comment this line for an assignment with lots of text

%----------------------------------------------------------------------------------------
%	TITLE SECTION
%----------------------------------------------------------------------------------------

\newcommand{\horrule}[1]{\rule{\linewidth}{#1}} % Create horizontal rule command with 1 argument of height

\title{	
\normalfont \normalsize 
\textsc{University of Virginia} \\ [25pt] % Your university, school and/or department name(s)
\horrule{0.5pt} \\[0.4cm] % Thin top horizontal rule
\huge CS 6161 Algorithms \\
\huge Homework 1 \\ % The assignment title
\horrule{2pt} \\[0.5cm] % Thick bottom horizontal rule
}

\author{Shawn (Shuoshuo) Chen\\sc7cq@virginia.edu\\Group partner: Rolph Recto\\ rjr7je@virginia.edu} % Your name

\date{\normalsize\today} % Today's date or a custom date

\begin{document}

\maketitle % Print the title

%----------------------------------------------------------------------------------------
%	PROBLEM 1
%----------------------------------------------------------------------------------------

\section*{Problem 1}
\textbf{(a)}. According to Tardos book, under the case that $k = n$, there is \\
\begin{align*} 
\begin{split}
Pr[X_i > c] &< \left( \frac{e^{c-1}}{c} \right)^c < \left( \frac{e}{c} \right)^c = \left( \frac{1}{\gamma(n)} \right)^{e\gamma(n)} < \left( \frac{1}{\gamma(n)} \right)^{2\gamma(n)} = \frac{1}{n^2}
\end{split}					
\end{align*}
By definition, $X_i$ is $|B(j)|$. Thus \\
\begin{align*} 
\begin{split}
Pr[ \ |B(j)| > c \ ] &< \frac{1}{n^2} \ \bigg|_{c \ = \ \Theta(\frac{log(n)}{loglog(n)})}
\end{split}					
\end{align*}
Applying the union bound over this upper bound for $|B(1)|, |B(2)|, ..., |B(n)|$, there is following \\
\begin{align*} 
\begin{split}
Pr[ \ |B(*)| > c \ ] &< \frac{1}{n} \ \bigg|_{c \ = \ \Theta(\frac{log(n)}{loglog(n)})}
\end{split}					
\end{align*}
In words, it means that with probability at least $1 - \frac{1}{n}$, $max_j\{|B(j)|\}$ will not exceed
$\Theta(\frac{log(n)}{loglog(n)})$.
\\

\textbf{(b)}. By definition, $|B(i)| = Y = \sum_{j \in [n]} Y_j$, thus $Var[|B(i)|] = Var[Y]$. And there is \\
\begin{align*} 
\begin{split}
Var[Y] &= \mathbb{E}[Y^2] - (\mathbb{E}[Y])^2 \\
&= \mathbb{E}[(\sum_{j \in [n]} Y_j)^2] - (\mathbb{E}[\sum_{j \in [n]} Y_j])^2 \\
&= \mathbb{E}[(\sum_{j \in [n]} Y_j)^2] - (\sum_{j \in [n]} \mathbb{E}[Y_j])^2 \\
&= \mathbb{E}[(\sum_{j \in [n]} Y_j)^2] - n \times \frac{1}{n} \\
&= \mathbb{E}[\sum_{j \in [n]}\sum_{i \in [n]} Y_iY_j] - 1 \\
&= \mathbb{E}[\sum_{j \in [n]} Y_i^2 + \sum_{j \in [n]}\sum_{i \in [n], i \neq j} Y_iY_j] - 1 \\
&= \sum_{j \in [n]}\mathbb{E}[Y_i^2] + \sum_{j \in [n]}\sum_{i \in [n], i \neq j}\mathbb{E}[Y_iY_j] -1 \\
&= \sum_{j \in [n]}\mathbb{E}[Y_i] + \sum_{j \in [n]}\sum_{i \in [n], i \neq j}\mathbb{E}[Y_i]\mathbb{E}[Y_j] -1 \\
&= n \times \frac{1}{n} + \frac{n(n-1)}{2} \times 2 \times \frac{1}{n^2} -1 \\
&= \frac{2n-1}{n} \\
&= 2 - \frac{1}{n}
\end{split}					
\end{align*}
As we can see, with $n \to \infty$, $Var[Y] \to 2$. And for any $n$, $Var[Y] \leqslant 2$. \\
That is, $Var[|B(i)|] = O(1)$.
\\

\textbf{(c)}. The Chebyshev's inequality gives
\begin{align*} 
\begin{split}
Pr[|X-\mu| \geqslant k\delta] \leqslant \frac{1}{k^2}
\end{split}					
\end{align*}
We can replace $X = |B(j)|$, $\mu = 1$ and $\delta = \sqrt{\frac{2n-1}{n}}$.
\begin{align*} 
\begin{split}
&Pr[|\ |B(j)|-1\ | \geqslant k\sqrt{\frac{2n-1}{n}}] \leqslant \frac{1}{k^2} \\
&Pr[|B(j)| \geqslant 1+k\sqrt{\frac{2n-1}{n}}] \leqslant \frac{1}{k^2} \\
\end{split}					
\end{align*}
If we make $k^2 = 100n$, solve for k, we get $k = 10\sqrt{n}$.
Replace $k = 10\sqrt{n}$:
\begin{align*} 
\begin{split}
Pr[|B(j)| \geqslant 1+10\sqrt{2n-1}] \leqslant \frac{1}{100n}
\end{split}					
\end{align*}
And it is easy to see that:
\begin{align*} 
\begin{split}
Pr[|B(j)| \textgreater c\sqrt{n}] \leqslant \frac{1}{100n} \ \bigg|_{c\sqrt{n} \ \textgreater \ 1+10\sqrt{2n-1}}
\end{split}					
\end{align*}
Solve for c in $c\sqrt{n} \ \textgreater \ 1+10\sqrt{2n-1}$, we get $c \textgreater \frac{1}{\sqrt{n}} + 10\sqrt{\frac{2n-1}{n}}$. \\
Note that $c > 10\sqrt{2} \ \bigg|_{n \to \infty}$. \\
To sum up, for every fixed $j \in [k=n]$, there is
\begin{align*} 
\begin{split}
Pr[|B(j)| \textgreater c\sqrt{n}] \leqslant \frac{1}{100n} \ \bigg|_{c \ \textgreater \ 10\sqrt{2}}
\end{split}					
\end{align*}
Applying the union bound over this upper bound for $|B(1)|, |B(2)|, ..., |B(n)|$, we get
\begin{align*} 
\begin{split}
Pr[\forall j \in [n]: |B(j)| \textgreater c\sqrt{n}] \leqslant \frac{1}{100} \ \bigg|_{c \ \textgreater \ 10\sqrt{2}}
\end{split}					
\end{align*}
Flip it over, we have
\begin{align*} 
\begin{split}
Pr[\forall j \in [n]: |B(j)| \leqslant c\sqrt{n}] \geqslant \frac{99}{100} \ \bigg|_{c \ \textgreater \ 10\sqrt{2}}
\end{split}					
\end{align*}
\\


%----------------------------------------------------------------------------------------
%	PROBLEM 2
%----------------------------------------------------------------------------------------

\section*{Problem 2}
\textbf{(a)}. Since $a, b, x \in \mathbb{Z}_p$, $h_{a,b}(x) = ax+b = z \in \mathbb{Z}_p$
\begin{align*} 
\begin{split}
Pr[h_{a,b}(x) = z] &= Pr[ax+b = z \ mod \ p] \\
&= Pr[ax \ mod \ p = z-b \ mod \ p] \\
&= Pr[a^{-1}ax \ mod \ p = a^{-1}(z-b) \ mod \ p] \\
&= Pr[x \ mod \ p = a^{-1}(z-b) \ mod \ p] \\
&= \frac{1}{p}
\end{split}					
\end{align*}
As above shows, the probability of getting z is identical to the probability of uniformly randomly picking x, which is uniform over $\mathbb{Z}_p$. \\
Then, for $x \neq y$,
\begin{align*} 
\begin{split}
Pr[h_{a,b}(x) = k \land h_{a,b}(y) = m] &= Pr[ax+b=k \ \land \ ay+b=m \  (mod\ p)] \\
&= Pr[x=a^{-1}(k-b) \ \land \ y=a^{-1}(m-b) \  (mod\ p)] \\
&= Pr[x=a^{-1}(k-b) \  (mod\ p)] \times Pr[y=a^{-1}(m-b) \ (mod\ p)] \\
&= \frac{1}{p} \times \frac{1}{p} \\
&= \frac{1}{p^2}
\end{split}					
\end{align*}
Thus, $h_{a,b}(x)$ and $h_{a,b}(y)$ are pairwise independent for $x \neq y$. \\
And $H=\{ h_{a,b}(x)=ax+b \ mod\ p\ |\ a,b \in \mathbb{Z}_p \}$ is a pairwise uniform family from $\mathbb{Z}_p$ to itself. \\

\textbf{(b)}. Given the statement in part b, the multiplicative inverse still exists under any general finite field $\{0,1\}^k$. \\
In a similar way, first sample $l$ from $\{0,1\}^{2k}$, where $l$ has $2k$ bits. Then slice $l$ into two pieces.
Assign ($1$st bit, $k$th bit) to $a$ and ($(k+1)$th bit, $(2k)$th bit) to $b$ such that $a,b \in \{0,1\}^k$. \\
We can still construct a family of hash functions $H=\{ h_l(x)=ax+b \ mod \ 2^k\ |\ a,b \in \{0,1\}^k \}$. \\
For any $x \in \{0,1\}^k$:
\begin{align*} 
\begin{split}
Pr[h_l(x) = z] &= Pr[ax+b = z \ mod \ 2^k] \\
&= Pr[ax \ mod \ 2^k = z-b \ mod \ 2^k] \\
&= Pr[a^{-1}ax \ mod \ 2^k = a^{-1}(z-b) \ mod \ 2^k] \\
&= Pr[x \ mod \ 2^k = a^{-1}(z-b) \ mod \ 2^k] \\
&= \frac{1}{2^k}
\end{split}					
\end{align*}
The probability of getting z is identical to the probability of uniformly randomly picking x, which is uniform over
$\{0,1\}^k$. \\
Then, for $x \neq x'$,
\begin{align*} 
\begin{split}
Pr[h_l(x) = k \land h_l(x') = m] &= Pr[ax+b=k \ \land \ ax'+b=m \  (mod\ 2^k)] \\
&= Pr[x=a^{-1}(k-b) \ \land \ x'=a^{-1}(m-b) \  (mod\ 2^k)] \\
&= Pr[x=a^{-1}(k-b) \  (mod\ 2^k)] \times Pr[x'=a^{-1}(m-b) \ (mod\ 2^k)] \\
&= \frac{1}{2^k} \times \frac{1}{2^k} \\
&= \frac{1}{2^{2k}}
\end{split}					
\end{align*}
Thus, $h_l(x)$ and $h_l(x')$ are pairwise independent for $x \neq x'$. \\
And $H=\{ h_l(x)=ax+b \ mod\ 2^k\ |\ a,b \in \{0,1\}^k \}$ is a pairwise uniform family from $\{0,1\}^k$ to itself. \\

\textbf{(c)}. For the case $m \textless n = k$, sample $l$ from $\{0,1\}^{2k}$, where $l$ has $2k=2n$ bits. Then slice $l$ into two pieces.
Assign ($1$st bit, $n$th bit) to $a$ and ($(n+1)$th bit, $(2n)$th bit) to $b$ such that $a,b \in \{0,1\}^n$. \\
We can still construct a family of hash functions $H=\{ h_l(x)=ax+b \ mod \ 2^n\ |\ a,b \in \{0,1\}^n \}$. \\
Note that $ax \in \{0,1\}^{m+n}$ and thus $ax+b \in \{0,1\}^{m+n}$ where
$m+n \textgreater n$. The input $x \in \{0,1\}^m$ is automatically padded to $\{0,1\}^{m+n}$. After the $mod \ 2^n$, the output is mapped into $\{0,1\}^n$. \\

For any $x,x' \in \{0,1\}^m, x \neq x'$,
\begin{align*} 
\begin{split}
Pr[h_l(x) = r \land h_l(x') = s] &= Pr[ax+b=r \ \land \ ax'+b=s \  (mod\ 2^n)] \\
&= Pr[a=\frac{1}{(x-x')}r - \frac{1}{(x-x')}s\ \land \ b=\frac{-x}{(x-x')}r + \frac{x'}{(x-x')}s\  (mod\ 2^n)] \\
&= Pr[a=\frac{1}{(x-x')}r - \frac{1}{(x-x')}s\  (mod\ 2^n)] \\
&\quad \times Pr[b=\frac{-x}{(x-x')}r + \frac{x'}{(x-x')}s\  (mod\ 2^n)] \\
&= \frac{1}{2^n} \times \frac{1}{2^n} \\
&= \frac{1}{2^{2n}}
\end{split}					
\end{align*}
Thus, $h_l(x)$ and $h_l(x')$ are pairwise independent for $x \neq x'$. And $H=\{ h_l(x)=ax+b \ mod\ 2^n\ |\ a,b \in \{0,1\}^n \}$ is a pairwise independent family from $\{0,1\}^m$ to $\{0,1\}^n$. \\

For the case $n \textless m = k$, sample $l$ from $\{0,1\}^{2k}$, where $l$ has $2k=2m$ bits. Then slice $l$ into two pieces.
Assign ($1$st bit, $m$th bit) to $a$ and ($(m+1)$th bit, $(2m)$th bit) to $b$ such that $a,b \in \{0,1\}^m$. \\
We can still construct a family of hash functions $H=\{ h_l(x)=ax+b \ mod \ 2^n\ |\ a,b \in \{0,1\}^m \}$. \\
Note that $ax \in \{0,1\}^{2m}$ and thus $ax+b \in \{0,1\}^{2m}$ where
$2m \textgreater n$. The input $x \in \{0,1\}^m$ shrinks to $\{0,1\}^n$ after the $mod \ 2^n$. \\

For any $x,x' \in \{0,1\}^m, x \neq x'$,
\begin{align*} 
\begin{split}
Pr[h_l(x) = r \land h_l(x') = s] &= Pr[ax+b=r \ \land \ ax'+b=s \  (mod\ 2^n)] \\
&= Pr[ax+b=r \  (mod\ 2^n)] \times Pr[ax'+b=s \ (mod\ 2^n)] \\
&= \frac{1}{2^n} \times \frac{1}{2^n} \\
&= \frac{1}{2^{2n}}
\end{split}					
\end{align*}
Thus, $h_l(x)$ and $h_l(x')$ are pairwise independent for $x \neq x'$. And $H=\{ h_l(x)=ax+b \ mod\ 2^n\ |\ a,b \in \{0,1\}^m \}$ is a pairwise independent family from $\{0,1\}^m$ to $\{0,1\}^n$. \\
\\


%----------------------------------------------------------------------------------------
%	PROBLEM 3
%----------------------------------------------------------------------------------------

\section*{Problem 3}
\textbf{(a)}. Given the fact that the communication pattern of the protocol is always fixed,
both Alice and Bob feed their input and previous communication history into the protocol and get the message
to communicate next, which is:
\begin{align*} 
\begin{split}
m_i = P(x, \sum_{j=0}^{i-1} m_{j})
\end{split}					
\end{align*}
Assume the communication starts from Alice. For $m_0$, Alice has no history but just uses her database $x$.
Then Bob will use $m_0$ and his database $y$ to generate $m_1$. And Alice sends $m_2$ and so on so forth. \\

If we know $c=c'$, which indicates $(m_0,m_1,...,m_t)=(m'_0,m'_1,...,m'_t)$, by induction we can see that
$\forall i \leqslant n-1: m_i=m'_i$. No matter Alice uses $x$ or $x'$, $m_0=m'_0$. This actually implies that $x=x'$. Similarly, Bob also has $y=y'$. \\

As a conclusion, there is $c_{x,y}=c_{x',y'}=c_{x',y}=c_{x,y'}$. \\


\textbf{(b)}. For a fixed input $x \in \{0,1\}^n$, we know that protocol $\Pi$ has a fixed communication pattern.
Assume protocol $\Pi$ takes $m$ bits of interaction where $m \textless n$.
Then we can always repeatedly run protocol $\Pi$ to figure out which $m \in x$ bits are taken for communication. \\

Next, based on $x$, we are able to purposely construct a $y$ such that $x \neq y$ but the $m$ bits from
$x$ and $y$ are the same. \\

According to part a, if $c_x=Enc(x,x)=c_y=Enc(y,y)$, there is also $c_{xy}=Enc(x,y)=c_{yx}=Enc(y,x)=c_x=c_y$. \\

Alice and Bob will end up conclude that even for pairs $(x,y)$ and $(y,x)$, their inputs equal (which is wrong).
\\


%----------------------------------------------------------------------------------------
%	PROBLEM 4
%----------------------------------------------------------------------------------------

\section*{Problem 4}
First, to prove $P_A(M_1)P_A(M_2) \equiv P_A(M_1M_2) \quad (mod n)$.
\begin{align*} 
\begin{split}
P_A(M_1)P_A(M_2) &= (M_1^e \  mod \ n )(M_2^e \  mod \ n ) \\
&= (M_1^eM_2^e) \ mod \ n \\
&= (M_1M_2)^e \ mod \ n \\
&= P_A(M_1M_2)
\end{split}					
\end{align*}
Assume the message we want to crack is $M$, which is encrypted in $P_A(M)$.
We can pick a message $M_r \in \mathbb{Z}_n$ at random and make sure that $M_{r}^{-1} \ mod \ n$ exists.
Since $M_r \in \mathbb{Z}_n$, there should exist its unique multiplicative inverse $M_r^{-1}$. \\

Then encrypt $M_r$ to get $P_A(M_r)$. And use what we have proved to get
$P_A(MM_r) = P_A(M)P_A(M_r)$. \\

If an advesary can decrypt the messages from $\mathbb{Z}_n$ by $1\%$, it means the probability of success
is $p = 0.01$ to decrypt $P_A(MM_r)$ and get $MM_r \ mod \ n$. If the decryption attempt is successful,
we can thus compute $M_r^{-1} \ mod \ n$ and derive $M = (MM_r)M_r^{-1} \ mod \ n$. \\

Now let us discuss the probability of successful decryption. It is noted that the probability of successful decryption
for a single attempt is $p = 0.01$. Thus the failing probability for a single attempt is $1-p = 1- 0.01 = 0.99$.
We want to make the overall probability high when decrypting $M$, which implies we need to keep trying.
Assume we need to try $t$ times to decrypt $M$, the probability of success in $t$ times is
$P_{decrypt} = 1-0.99^t$. If "high probability" means $99\%$, solve for $t$:
\begin{align*} 
\begin{split}
P_{decrypt} &= 1 - 0.99^t \\
&= 0.99 \\
\end{split}					
\end{align*}
\begin{align*} 
\begin{split}
t &= \log_{0.99}(0.01) \\
&= 458.2
\end{split}					
\end{align*}
Thus we get $t = 459$.
\\


\end{document}
