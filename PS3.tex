%%%%%%%%%%%%%%%%%%%%%%%%%%%%%%%%%%%%%%%%%
% Short Sectioned Assignment
% LaTeX Template
% Version 1.0 (5/5/12)
%
% This template has been downloaded from:
% http://www.LaTeXTemplates.com
%
% Original author:
% Frits Wenneker (http://www.howtotex.com)
%
% License:
% CC BY-NC-SA 3.0 (http://creativecommons.org/licenses/by-nc-sa/3.0/)
%
%%%%%%%%%%%%%%%%%%%%%%%%%%%%%%%%%%%%%%%%%

%----------------------------------------------------------------------------------------
%	PACKAGES AND OTHER DOCUMENT CONFIGURATIONS
%----------------------------------------------------------------------------------------

\documentclass[titlepage, paper=a4, fontsize=11pt]{scrartcl} % A4 paper and 11pt font size

\usepackage[T1]{fontenc} % Use 8-bit encoding that has 256 glyphs
\usepackage{fourier} % Use the Adobe Utopia font for the document - comment this line to return to the LaTeX default
\usepackage[english]{babel} % English language/hyphenation
\usepackage{amsmath,amsfonts,amsthm} % Math packages
\usepackage{listings}

\usepackage{lipsum} % Used for inserting dummy 'Lorem ipsum' text into the template


\usepackage{sectsty} % Allows customizing section commands
\allsectionsfont{\centering \normalfont\scshape} % Make all sections centered, the default font and small caps

\usepackage{fancyhdr} % Custom headers and footers
\pagestyle{fancyplain} % Makes all pages in the document conform to the custom headers and footers
\fancyhead{} % No page header - if you want one, create it in the same way as the footers below
\fancyfoot[L]{} % Empty left footer
\fancyfoot[C]{} % Empty center footer
\fancyfoot[R]{\thepage} % Page numbering for right footer
\renewcommand{\headrulewidth}{0pt} % Remove header underlines
\renewcommand{\footrulewidth}{0pt} % Remove footer underlines
\setlength{\headheight}{13.6pt} % Customize the height of the header

\numberwithin{equation}{section} % Number equations within sections (i.e. 1.1, 1.2, 2.1, 2.2 instead of 1, 2, 3, 4)
\numberwithin{figure}{section} % Number figures within sections (i.e. 1.1, 1.2, 2.1, 2.2 instead of 1, 2, 3, 4)
\numberwithin{table}{section} % Number tables within sections (i.e. 1.1, 1.2, 2.1, 2.2 instead of 1, 2, 3, 4)

\setlength\parindent{0pt} % Removes all indentation from paragraphs - comment this line for an assignment with lots of text

%----------------------------------------------------------------------------------------
%	TITLE SECTION
%----------------------------------------------------------------------------------------

\newcommand{\horrule}[1]{\rule{\linewidth}{#1}} % Create horizontal rule command with 1 argument of height

\title{	
\normalfont \normalsize 
\textsc{University of Virginia} \\ [25pt] % Your university, school and/or department name(s)
\horrule{0.5pt} \\[0.4cm] % Thin top horizontal rule
\huge CS 6161 Algorithms \\
\huge Problem Set 3 \\ % The assignment title
\horrule{2pt} \\[0.5cm] % Thick bottom horizontal rule
}

\author{Shawn (Shuoshuo) Chen\\sc7cq@virginia.edu\\Group partner: Rolph Recto\\ rjr7je@virginia.edu} % Your name

\date{\normalsize\today} % Today's date or a custom date

\begin{document}

\maketitle % Print the title

%----------------------------------------------------------------------------------------
%	PROBLEM 1
%----------------------------------------------------------------------------------------

\section*{Problem 1}
\textbf{(a).}
Assuming the value of ${x_1,x_2, ... ,x_n}$ is pick at random independently. Then the probability of getting true
and false is equal, where $p_{true} = p_{false} = 0.5$. For a clause $C_i$, the probability of evaluating true is
$E[p_{true}] = 1-2^{-d}$. And there are $m$ clauses in $\phi$, so the number of satified clauses is
$E[N_{SAT}] = (1-2^{-d}) * m$, which indicates there is an assignment satisfying at least $(1-2^{-d})m$ clauses.
\\

\textbf{(b).}
Assume $\forall \epsilon \in (0,1), \exists \delta \in (0,1): Pr[K \geq (1-2^{-d}-\epsilon)m] \textgreater \delta$.
Let $K$ be the number of satisfied clauses, there is:
\begin{align*} 
\begin{split}
\mathbb{E}[K] &= \sum_{k} kPr[K=k] \\
&= \sum_{k \textless (1-2^{-d}-\epsilon)m} kPr[K=k] + \sum_{k \geq (1-2^{-d}-\epsilon)m} kPr[K=k] \\
&\textless (1-2^{-d}-\epsilon)mPr[K \textless (1-2^{-d}-\epsilon)m] + mPr[K \geq (1-2^{-d}-\epsilon)m] \\
&= (1-2^{-d}-\epsilon)m + (2^{-d}+\epsilon)mPr[K \geq (1-2^{-d}-\epsilon)m] \\
\end{split}
\end{align*}
And recall $\mathbb{E}[K]=1-2^{-d}$, we can derive:
\begin{align*} 
\begin{split}
Pr[K \geq (1-2^{-d}-\epsilon)m] \textgreater \frac{\epsilon}{2^{-d}+\epsilon}
\end{split}
\end{align*}
So let $\delta = \frac{\epsilon}{2^{-d}+\epsilon}$, we essentially have $Pr[K \geq (1-2^{-d}-\epsilon)m] \textgreater \delta$.
With probability 0.99, we want to find an assignment that satisfies. In other words, we need to make $(1-\delta)^t  \leq 0.01$. Solve for t, we can get the number of times to run this algorithm to be $t \geq log_{1-\delta} 0.01 = log_{\frac{2^{-d}}{2^{-d}+0.0001}} 0.01$.
\\



%----------------------------------------------------------------------------------------
%	PROBLEM 2
%----------------------------------------------------------------------------------------

\section*{Problem 2}
\textbf{(a).}
There are 2 cases: 4 nodes in 2 edges and 3 nodes in 2 edges. \\

For the case 4 nodes in 2 egdes, there are $X_i, X_j, X_i', X_j'$:
\begin{align*} 
\begin{split}
Pr[Y_e=1 \land Y_e'=1] &= \frac{4}{16}
\end{split}
\end{align*}
where the satisfied sequence of $(X_i, X_j, X_i', X_j')$ is 1010, 1001, 0110, 0101.
Also, there is:
\begin{align*} 
\begin{split}
Pr[Y_e=1] &= Pr[Y_e'=1] = \frac{1}{2}
\end{split}
\end{align*}
Thus:
\begin{align*} 
\begin{split}
Pr[Y_e=1 \land Y_e'=1] &= Pr[Y_e=1]Pr[Y_e']
\end{split}
\end{align*}
Similarly, for the 3-node case, there are $X_i, X_j, X_j'$:
\begin{align*} 
\begin{split}
Pr[Y_e=0 \land Y_e'=1] &= \frac{2}{8}
\end{split}
\end{align*}
where the satisfied sequence of $(X_i, X_j, X_j')$ is 001, 110.
Thus:
\begin{align*} 
\begin{split}
Pr[Y_e=1 \land Y_e'=1] &= Pr[Y_e=1]Pr[Y_e'] \\
&= \frac{1}{2} * \frac{1}{2} = \frac{1}{4}
\end{split}
\end{align*}
That $e$ and $e'$ are independent for $e \neq e'$.
\\

\textbf{(b).}
If $e_1,e_2, ... , e_k$ form a cycle, let us assume that we already know the first $k-1$ edges are in the same partition, then the last edge is actually fixed since both of its endpoints are already fixed by $e_1$ and $e_{k-1}$.
Thus:
\begin{align*} 
\begin{split}
Pr[Y_{e_1}=0 \land Y_{e_2}=0 \land ... \land Y_{e_k}=0] &= Pr[Y_{e_1}=0 \land Y_{e_2}=0 \land ... \land Y_{e_{k-1}}=0] \\
&= Pr[Y_{e_1}=0]Pr[Y_{e_2}=0]...Pr[Y_{e_{k-1}}=0] \\
&\neq Pr[Y_{e_1}=0]Pr[Y_{e_2}=0]...Pr[Y_{e_{k-1}}=0] * \frac{1}{2} \\
&= Pr[Y_{e_1}=0]Pr[Y_{e_2}=0]...Pr[Y_{e_{k}}=0]
\end{split}
\end{align*}
Thus $Y_{e_1},Y_{e_2}, ... , Y_{e_k}$ are not independent.



%----------------------------------------------------------------------------------------
%	PROBLEM 3
%----------------------------------------------------------------------------------------

\section*{Problem 3}
\textbf{(a).}
According to the definition of expectation, for a random variable $X: \Omega \to \mathbb{Z}$, there is:
\begin{align*} 
\begin{split}
\mathbb{E}[X] &= \sum_{\omega \in \Omega} X(\omega)Pr[\omega] \\
\end{split}					
\end{align*}
So for the claim:
\begin{align*} 
\begin{split}
\mathbb{E}[\sum_{i} X_i] &= \sum_{\omega \in \Omega} \sum_{i} X_i(\omega)Pr[\omega] \\
&= \sum_{i} \sum_{\omega \in \Omega} X_i(\omega)Pr[\omega] \\
&= \sum_{i} \mathbb{E}[X_i]
\end{split}					
\end{align*}
\\

\textbf{(b).}
Imagine that $\{X_i\}$ is somehow controlled by an invisible hand such that
$\forall X_i \in \{X_i\}$, it becomes the same value as others.
In other words, $X_1=X_2= ... =X_i$ at the same time.
Now assume $X_i$ is either $0$ or $1$. Essentially, $X_1X_2...X_i$ is either $0$ or $1$.
Thus, there is:
\begin{align*} 
\begin{split}
\mathbb{E}[\prod X_i] &= \mathbb{E}[X_1X_2...X_i] = \frac{1}{2} \\
&\neq \prod_i \mathbb{E}[X_i] = \prod_i \frac{1}{2} = \frac{1}{2^i}
\end{split}					
\end{align*}
\\

\textbf{(c).}
If $\{X_i\}$ is independent, $Pr[X_1X_2...X_i]=Pr[X_1]Pr[X_2]...Pr[X_i]$.
Then,
\begin{align*} 
\begin{split}
\mathbb{E}[\prod X_i] &= \mathbb{E}[X_1X_2...X_i] \\
&= \sum_{X_1X_2...X_i} X_1X_2...X_iPr[X_1=x_1,X_2=x_2, ... ,X_i=x_i] \\
&= \sum_{X_1}\sum_{X_2}...\sum_{X_i} x_1x_2...x_iPr[X_1=x_1,X_2=x_2, ... ,X_i=x_i] \\
&= \sum_{X_1}\sum_{X_2}...\sum_{X_i} x_1x_2...x_iPr[X_1=x_1]Pr[X_2=x_2]...Pr[X_i=x_i] \\
&= \sum_{X_1}x_1Pr[X_1=x_1]\sum_{X_2}x_2Pr[X_2=x_2]...\sum_{X_i}x_iPr[X_i=x_i] \\
&= \mathbb{E}[X_1]\mathbb{E}[X_2]...\mathbb{E}[X_i] \\
&= \prod_i \mathbb{E}[X_i]
\end{split}					
\end{align*}
\\



%----------------------------------------------------------------------------------------
%	PROBLEM 4
%----------------------------------------------------------------------------------------

\section*{Problem 4}
\textbf{18.16 (a).}
According to $D = P(H) * E[D | H] + P(T) * E[D | T] = H * D + T * (H * D + T)$ where $D$ is the state of both the start point and end point of H and TH in a tree diagram, there is:
\begin{align*} 
\begin{split}
E[N_{TT}] &= 0.5 * E[N_{TT} | H] + 0.5 * E[N_{TT} | T] \\
&= 0.5 * (1 + N_{TT}) + 0.5 * [1 + E[N_{TT} | 2nd]] \\
&= 0.5 * (1 + N_{TT}) + 0.5 * [1 + 0.5 * 1 + 0.5 * (1 + N_{TT})]
\end{split}					
\end{align*}
Solve for $N_{TT}$ and we get $N_{TT} = 6$.
Note that if we get a head (H), it basically adds one toss to the expectation. On the other hand, if we get a tail (T) followed by a head (H), it adds one toss to $E[N_{T*}]$, where $E[N_{T*}]$ is a combination of TT and TH. If we reach TT, then that's the end, just add another toss. Otherewise we need to start over again, so just plug the same structure $1 + N_{TT}$ in.
\\
\textbf{18.16 (b).}
Similarly, there is $C = H * C + T * B$ where $B = H + T * B$ for the tree diagram. We can derive:
\begin{align*} 
\begin{split}
E[N_{TH}] &= 0.5 * ( 1 + E[N_{TH} | H] ) + 0.5 * E[N_{B} | T] \\
\end{split}					
\end{align*}
And for $E_{B}$, there is:
\begin{align*} 
\begin{split}
E[N_{B}] &= 0.5 * 1 + 0.5 * ( 1 + E[N_{B}] ) \\
\end{split}					
\end{align*}
Solve for $N_{B}$ we get $N_{B} = 2$, and plug it into the $N_{TH}$ equation we get $N_{TH} = 4$.
\\
\textbf{18.16 (c).}
Since TT and TH both have the same first toss T, and by probability, the chance of getting T or H in the second toss is equally half. So $P(TT) = P(TH) = 0.5$. Then we can get the expected profit:
\begin{align*} 
\begin{split}
E[Profit] &= 0.5 * 6 + 0.5 * (-5) \\
&= 0.5 \quad dollars
\end{split}					
\end{align*}
\\

\textbf{18.19}
We can come up with a tree diagram with 3 levels. And there is:
\begin{align*} 
\begin{split}
E[N_{TTH}] &= p * (1+N_{TTH}) + (1-p) * (1+E[N_C]) \quad where \\
E[N_C] &= p * (1+N_{TTH}) + (1-p) * (1+E[N_B]) \quad where \\
E[N_B] &= p * 1 + (1-p) * (1+E[N_B])
\end{split}					
\end{align*}
First solve for $E[N_B]$, we get $E[N_B] = \frac{1}{p}$. Then plug $E[N_B]$ into $E[N_C]$, we get
$E[N_C] = \frac{1}{p} + p * E[N_{TTH}]$. Finally, plug $E[N_C]$ into $E[N_{TTH}]$, we get $E[N_{TTH}] = \frac{1}{p(p-1)^2}$. \\
To minimize $E[N_{TTH}]$, we differentiate it with respect to $p$:
\begin{align*} 
\begin{split}
d(E[N_{TTH}])/dp &= \frac{1-3p}{(p-1)^3p^2} \\
\end{split}					
\end{align*}
So when $p=\frac{1}{3}$, $E[N_{TTH}]$ becomes minimum.
\\


%----------------------------------------------------------------------------------------
%	PROBLEM 5
%----------------------------------------------------------------------------------------

\section*{Problem 5}
\textbf{1.}
Let us use the same notation from the class, where $m$ is the number of edges in $G$.
Then we can see $c^* \leq m$. \\

Now, we randomly color each node with one of the three colors. Since the nodes are uniformly distributed and pairwise independent, the probability of getting each color is equally $\frac{1}{3}$. Then we can use the similar trick: \\
Let $Y_e$ be a random variable that indicates whether an edge $e$ is satified or not:
\begin{gather*}
Y_e =
\begin{cases}
1 \quad satisfied \\
0 \quad not\ satisfied \\
\end{cases}
\end{gather*}
It is easy to see that there are $3 \time 3 = 9$ combinations or coloring, where $3$ of them do not satisfy. So:
\begin{align*} 
\begin{split}
\mathbb{E}[Y_e]=\frac{2}{3}
\end{split}					
\end{align*}
And let $Y$ be a random variable indicating the total number of satisfied edges:
\begin{align*} 
\begin{split}
Y = \sum_{e \in [m]} Y_e
\end{split}					
\end{align*}
There is:
\begin{align*} 
\begin{split}
\mathbb{E}[Y] &= \mathbb{E}[\sum_{e \in [m]} Y_e] \\
&= \sum_{e \in [m]} \mathbb{E}[Y_e] \\
&= \frac{2}{3}m \geq \frac{2}{3}c^*
\end{split}					
\end{align*}
\\

\textbf{8.}
First, we just uniformly pick $k \in [n]$ nodes at random. And let
\begin{gather*}
Y_{uv} =
\begin{cases}
1 \quad exist\ an\ edge\ between\ (u,v) \\
0 \quad not\ exist \\
\end{cases}
\end{gather*}
Note that there will be $2m$ edges in total because (u,v) and (v,u) are doubly counted. \\
Then we can get:
\begin{align*} 
\begin{split}
\mathbb{E}[Y_{uv}] &= \frac{2m}{n(n-1)} \\
\end{split}					
\end{align*}
Note that u and v cannot be the same node, otherwise it is not an edge. \\
Then in $G[k]$, the expected number of edges is:
\begin{align*} 
\begin{split}
\mathbb{E}[G[k]_{edge}] &= \mathbb{E}[\sum_{k \in [n]} Y_[uv]] \\
&= \sum_{k \in [n]} \mathbb{E}[Y_{uv}] \\
&= \frac{k(k-1)m}{n(n-1)}
\end{split}					
\end{align*}
Now assume $e=\mathbb{E}[G[k]_{edge}]$, its upper bound is $e_m = \frac{k(k-1)}{2}$ (a full graph).
We can repeatedly run this randomized algorithm multiple times until we get the expected value of edges. The probability of getting $e$ in a single run is $p$. And the probability of finding $j$ edges is $p_j$.
So there is:
\begin{align*} 
\begin{split}
e &= \sum_{j}jp_j = \sum_{j \textless e}jp_j + \sum_{j \geq e}jp_j \\
&\textless \sum_{j \textless e}(e - \epsilon)p_j + \sum_{j \geq e}(e_m)p_j \\
&= (e-\epsilon)(1-p) + \frac{k(k-1)}{2}p \\
&\leq e - \epsilon + \frac{k(k-1)}{2}p
\end{split}					
\end{align*}
We can see that $p \geq \frac{2\epsilon}{k(k-1)}$. Since the randomized algorithm itself is polynomial time, and repeating it $\frac{1}{p}$ times is still polynomial. Thus, we can just run this algorithm $\frac{1}{p}$ times to get the expected edges.
\\


%----------------------------------------------------------------------------------------
%	PROBLEM 6
%----------------------------------------------------------------------------------------

\section*{Problem 6}
Since there is no bad node, according to the algorithm, nothing needs to be moved. \\
And previously, we have already proved in any graph, there will be at least half of the edges in the cut:
\begin{align*} 
\begin{split}
E[Y] &= E[\sum_{e \in [m]} Y_e] = \sum_{e \in [m]} E[Y_e] \\
&= \frac{1}{2} * m
\end{split}					
\end{align*}
And note there are $n$ nodes in the graph, so the maximum number of edges is $n(n-1)/2$. \\
If we assume all nodes are bad, according to the algorithm, we need to move every single node. The best result of that is we end up having all edges in the cut, which is $n(n-1)/2$ edges at most. \\
And for each move, the least effecient case is we only increase the number of edges by 1. In this case, we need to proceed $n(n-1)/2$ steps, and there is no worse case than it.


\end{document}
