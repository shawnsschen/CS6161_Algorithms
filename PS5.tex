%%%%%%%%%%%%%%%%%%%%%%%%%%%%%%%%%%%%%%%%%
% Short Sectioned Assignment
% LaTeX Template
% Version 1.0 (5/5/12)
%
% This template has been downloaded from:
% http://www.LaTeXTemplates.com
%
% Original author:
% Frits Wenneker (http://www.howtotex.com)
%
% License:
% CC BY-NC-SA 3.0 (http://creativecommons.org/licenses/by-nc-sa/3.0/)
%
%%%%%%%%%%%%%%%%%%%%%%%%%%%%%%%%%%%%%%%%%

%----------------------------------------------------------------------------------------
%	PACKAGES AND OTHER DOCUMENT CONFIGURATIONS
%----------------------------------------------------------------------------------------

\documentclass[titlepage, paper=a4, fontsize=11pt]{scrartcl} % A4 paper and 11pt font size

\usepackage[T1]{fontenc} % Use 8-bit encoding that has 256 glyphs
\usepackage{fourier} % Use the Adobe Utopia font for the document - comment this line to return to the LaTeX default
\usepackage[english]{babel} % English language/hyphenation
\usepackage{amsmath,amsfonts,amsthm} % Math packages
\usepackage{listings}

\usepackage{lipsum} % Used for inserting dummy 'Lorem ipsum' text into the template


\usepackage{sectsty} % Allows customizing section commands
\allsectionsfont{\centering \normalfont\scshape} % Make all sections centered, the default font and small caps

\usepackage{fancyhdr} % Custom headers and footers
\pagestyle{fancyplain} % Makes all pages in the document conform to the custom headers and footers
\fancyhead{} % No page header - if you want one, create it in the same way as the footers below
\fancyfoot[L]{} % Empty left footer
\fancyfoot[C]{} % Empty center footer
\fancyfoot[R]{\thepage} % Page numbering for right footer
\renewcommand{\headrulewidth}{0pt} % Remove header underlines
\renewcommand{\footrulewidth}{0pt} % Remove footer underlines
\setlength{\headheight}{13.6pt} % Customize the height of the header

\numberwithin{equation}{section} % Number equations within sections (i.e. 1.1, 1.2, 2.1, 2.2 instead of 1, 2, 3, 4)
\numberwithin{figure}{section} % Number figures within sections (i.e. 1.1, 1.2, 2.1, 2.2 instead of 1, 2, 3, 4)
\numberwithin{table}{section} % Number tables within sections (i.e. 1.1, 1.2, 2.1, 2.2 instead of 1, 2, 3, 4)

\setlength\parindent{0pt} % Removes all indentation from paragraphs - comment this line for an assignment with lots of text

%----------------------------------------------------------------------------------------
%	TITLE SECTION
%----------------------------------------------------------------------------------------

\newcommand{\horrule}[1]{\rule{\linewidth}{#1}} % Create horizontal rule command with 1 argument of height

\title{	
\normalfont \normalsize 
\textsc{University of Virginia} \\ [25pt] % Your university, school and/or department name(s)
\horrule{0.5pt} \\[0.4cm] % Thin top horizontal rule
\huge CS 6161 Algorithms \\
\huge Problem Set 5 \\ % The assignment title
\horrule{2pt} \\[0.5cm] % Thick bottom horizontal rule
}

\author{Shawn (Shuoshuo) Chen\\sc7cq@virginia.edu} % Your name

\date{\normalsize\today} % Today's date or a custom date

\begin{document}

\maketitle % Print the title

%----------------------------------------------------------------------------------------
%	PROBLEM 2
%----------------------------------------------------------------------------------------

\section*{Problem 2}
First, we prove (a) graph G is connected if it has n-1 edges and no cycle. \\
Assume there are k connected subgraphs in G. By the definition of tree, an undirected, connected, acyclic graph is a tree. Thus the k subgraphs are k trees. Thus, each subgraph has $|V_i|-1$ edges, which will be proved later. Then sum up all the edges in these k subgraphs, we get $|V|-k = n-k$ edges. According to property (b), there must be $k=1$, which effectively makes G connected as a whole component. \\

Next, we prove (b) graph G has n-1 edges if it is connected and acyclic. \\
We can prove it by showing $|E| \geq |V| -1$ and $|E| \leq |V| -1$. \\
By induction, if $|V|=1,2$, $|E| \geq 0,1$ is trivially true. Then assume it is also true for $|V| \leq n$. For a graph with $n+1$ vertices, choose a random vertex $v$, remove $v$ and its incident edges. Then the graph is divided into $k$ connected subgraphs with $n_1, n_2, ... , n_k$ vertices respectively. By assumption, there is $\sum_1^k |E_i| \geq \sum_1^k (n_i -1)=\sum_1^k n_i - k = n-k$. Now add $v$ back to G, since G is connected, there should be at least one edge between $v$ and each subgraph. Therefore,
$|E| \geq \sum_i^k |E_i| + k = n-k+k = n = |V| - 1$. \\
Similarly, it is trivial that a connected graph with 1 or 2 vertices has 0 or 1 edge. Assume $|E| \leq |V| -1$ for $|V| \leq n$. Removing an arbitary edge from G separates the graph into $k \geq 2$ connected subgraphs. Each subgraph, by definition, is a tree. Because each subgraph has less than $|V|=n$ vertices, it satisfies $|E_i| \leq |V_i| - 1$. Sum up all edges in all subgraphs, there is $\sum_1^k |E_i| \leq |V| - k \leq |V| - 2$. After adding the removed edge back, we have $|E| \leq |V| - 1$. \\
Therefore, we conclude $|E| = |V| - 1 = n-1$. \\

Finally, we prove (c) graph G has no cycle if it is connected and it has n-1 edges. \\
Suppose there is a cycle in G containing k vertices: $v_1, v_2, ... , v_k$. Let $G_k = (V_k, E_k)$ be the subgraph consisting of the cycle. Then there is $|V_k| = |E_k| = k$. If $k \textless |V|$, there must exist a vertex $v_{k+1} \in V - V_k$ that is adjacent to $v_i \in V_k$ due to G is connected. Let $G_{k+1} = (V_{k+1}, E_{k+1})$ be the subgraph with $V_{k+1} = V_k \cup \{v_{k+1}\}$ and $E_{k+1} = E_k \cup \{(v_i, v_{k+1})\}$. Note there is $|V_{k+1}| = |E_{k+1}| = k$. If $k+1 \textless |V|$, we can continue to $G_{k+2}$ until it reaches $G_n$ where $n=|V|, V_n=V, |E_n|=|V_n|=|V|$. Since $G_n$ is a subgraph of G, there is $|E| \geq |V|$. It contradicts the property that $|E|=|V|-1=n-1$. \\



%----------------------------------------------------------------------------------------
%	PROBLEM 4
%----------------------------------------------------------------------------------------

\section*{Problem 4}
First, we can call an implementation of Ford-Fulkerson algorithm, the Edmonds-Karp algorithm, as a subroutine to compute a min-cut according to the max-flow min-cut theorem. As per the Edmonds-Karp algorithm, its time complexity is $O(|V||E|^2)$. \\

The Ford-Fulkerson algorithm is actually to solve the directed graphs, but we can make our undirected graph directed by introducing two opposing directed edges on every undirected edge. This could essentially require $O(|V|^2)$ calls to the Edmonds-Karp algorithm. And some of the min-cut solutions could be equivalent for the original undirected graph. But effectively, this would not change the time bound that we will discuss later. \\

So we could repeatedly call Edmonds-Karp algorithm for each input $(s, t)$ and get a cut $C$ of volume $|C|$. Assume cut $C$ contains edge $e_1, e_2, ... , e_k$. For each $e_i: i \in [k]$, we increase the capacity of $e_i$ by a constant factor $\delta$, and call Edmonds-Karp again to get a new min-cut $C_i$ for it. When iterating over $e_i$, if there exists a $|C_i|=|C|$ and $C_i \neq C$, $C$ is not the unique min-cut. On the other hand, if there exists another min-cut $C'$ other than $C$, there also exists some $e_i \in C, e_i \notin C'$. Changing the capacity of that $e_i$ would not affect $C'$, which still leads to $|C|=|C_i|$. Thus, overall, the graph has a unique min-cut iff $|C| \textless |C_i|$ for all $i$. \\

To conclude, our algorithm is: \\
1. Turning undirected $G$ into a directed $G'$. \\
2. Enumerate all the possible $(s, t)$ in $G'$ and correspondingly call Edmonds-Karp algorithm. \\
3. After getting the cut $C$, iterate over all edges in $C$, increase capacity and call Edmonds-Karp. \\
4. See if $|C| \textless |C_i|$ for all $i$. If so, return $C$. \\

In the worst case, for each input $(s, t)$, we need to call Edmonds-Karp at most $|E|+1$ times to determine whether the first returned cut $C$ is unique or not. And for the undirected graph, we need to repeat $O(|V|^2)$ times. So eventually, this algorithm takes $O(|E|^3|V|^3)$, which is polynomial. \\



%----------------------------------------------------------------------------------------
%	PROBLEM 5
%----------------------------------------------------------------------------------------

\section*{Problem 5}
The algorithm is as follows: \\
1. Construct $G^3$ of graph $G = (V, E)$. \\
2. Compute the BST for $G^3$. \\
3. Take a full walk of the BST and log the first-time visited vertices sequentially in H. \\
4. return the Hamiltonian cycle H. \\

First, according to the definition of the 3rd power of graph G, $G^3 = (V, E^3)$ where there is an edge
$(u, v) \in E^3$ such that $\forall path_{uv} \in G$, it only contains at most 3 edges. In other words, if there exists an edge from u to v in $G^3$, the path from u to v in G would not exceed 3 edges. \\
Assume the costliest edge in the optimal Hamiltonian cycle ($H^*$) is $w^*$. Since the Halmitonian cycle is constructed by adding an edge to the BST, the costliest edge in the BST cannot exceed $w^*$. Because if the extra edge added into the BST is no heavier than $w^*$, the BST has the same cost as the cycle. Otherwise if the extra edge is heavier than $w^*$, the cycle has more cost than the BST. \\
Thus, in G, we have $c_{max}(BST^*) \leq c_{max}(H^*) = w^*$. \\
And also note that a full walk (W) of the BST satisfies $c_{max}(W) \leq 3w^* = 3c_{max}(H^*)$. Because by definition of $G^3$, the costliest edge in the walk can correspondingly map to a path in G with at most 3 edges. Even for the worst case, all 3 edges are equally costliest, which is $w^*$. The costliest edge in the walk will not exceed $3w^*$. \\
Finally, apply the triangle inequality, skipping nodes in the walk will not increase the cost, which leads to
$c_{max}(H) \leq c_{max}(W)$. \\
To sum it up, there is $c_{max}(H) \leq c_{max}(W) \leq 3c_{max}(H^*)$. \\
From above, we see that this algorithm gives an approximation ratio of 3. \\

Now consider the time complexity, for a graph with $|V|$ vertices, there could exist at most $\frac{|V|*(|V|-1)}{2}$ edges. Thus, constructing $G^3$ can be done within polynomial time $O(|V|^2)$. \\
And from problem 23-3, we can compute the BST by removing edges starting from the costliest in linear time. Taking a full walk can also easily be done in polynomial time. So overall, this algorithm is polynomial. \\




\end{document}
