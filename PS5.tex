%%%%%%%%%%%%%%%%%%%%%%%%%%%%%%%%%%%%%%%%%
% Short Sectioned Assignment
% LaTeX Template
% Version 1.0 (5/5/12)
%
% This template has been downloaded from:
% http://www.LaTeXTemplates.com
%
% Original author:
% Frits Wenneker (http://www.howtotex.com)
%
% License:
% CC BY-NC-SA 3.0 (http://creativecommons.org/licenses/by-nc-sa/3.0/)
%
%%%%%%%%%%%%%%%%%%%%%%%%%%%%%%%%%%%%%%%%%

%----------------------------------------------------------------------------------------
%	PACKAGES AND OTHER DOCUMENT CONFIGURATIONS
%----------------------------------------------------------------------------------------

\documentclass[titlepage, paper=a4, fontsize=11pt]{scrartcl} % A4 paper and 11pt font size

\usepackage[T1]{fontenc} % Use 8-bit encoding that has 256 glyphs
\usepackage{fourier} % Use the Adobe Utopia font for the document - comment this line to return to the LaTeX default
\usepackage[english]{babel} % English language/hyphenation
\usepackage{amsmath,amsfonts,amsthm} % Math packages
\usepackage{listings}

\usepackage{lipsum} % Used for inserting dummy 'Lorem ipsum' text into the template


\usepackage{sectsty} % Allows customizing section commands
\allsectionsfont{\centering \normalfont\scshape} % Make all sections centered, the default font and small caps

\usepackage{fancyhdr} % Custom headers and footers
\pagestyle{fancyplain} % Makes all pages in the document conform to the custom headers and footers
\fancyhead{} % No page header - if you want one, create it in the same way as the footers below
\fancyfoot[L]{} % Empty left footer
\fancyfoot[C]{} % Empty center footer
\fancyfoot[R]{\thepage} % Page numbering for right footer
\renewcommand{\headrulewidth}{0pt} % Remove header underlines
\renewcommand{\footrulewidth}{0pt} % Remove footer underlines
\setlength{\headheight}{13.6pt} % Customize the height of the header

\numberwithin{equation}{section} % Number equations within sections (i.e. 1.1, 1.2, 2.1, 2.2 instead of 1, 2, 3, 4)
\numberwithin{figure}{section} % Number figures within sections (i.e. 1.1, 1.2, 2.1, 2.2 instead of 1, 2, 3, 4)
\numberwithin{table}{section} % Number tables within sections (i.e. 1.1, 1.2, 2.1, 2.2 instead of 1, 2, 3, 4)

\setlength\parindent{0pt} % Removes all indentation from paragraphs - comment this line for an assignment with lots of text

%----------------------------------------------------------------------------------------
%	TITLE SECTION
%----------------------------------------------------------------------------------------

\newcommand{\horrule}[1]{\rule{\linewidth}{#1}} % Create horizontal rule command with 1 argument of height

\title{	
\normalfont \normalsize 
\textsc{University of Virginia} \\ [25pt] % Your university, school and/or department name(s)
\horrule{0.5pt} \\[0.4cm] % Thin top horizontal rule
\huge CS 6161 Algorithms \\
\huge Problem Set 5 \\ % The assignment title
\horrule{2pt} \\[0.5cm] % Thick bottom horizontal rule
}

\author{Shawn (Shuoshuo) Chen\\sc7cq@virginia.edu\\Group partner: Rolph Recto\\ rjr7je@virginia.edu} % Your name

\date{\normalsize\today} % Today's date or a custom date

\begin{document}

\maketitle % Print the title

%----------------------------------------------------------------------------------------
%	PROBLEM 5
%----------------------------------------------------------------------------------------

\section*{Problem 5}
The algorithm is as follows: \\
1. Construct $G^3$ of graph $G = (V, E)$. \\
2. Compute the BST for $G^3$. \\
3. Take a full walk of the BST and log the first-time visited vertices sequentially in H. \\
4. return the Hamiltonian cycle H. \\

First, according to the definition of the 3rd power of graph G, $G^3 = (V, E^3)$ where there is an edge
$(u, v) \in E^3$ such that $\forall path_{uv} \in G$, it only contains at most 3 edges. In other words, if there exists an edge from u to v in $G^3$, the path from u to v in G would not exceed 3 edges. \\
Assume the costliest edge in the optimal Hamiltonian cycle ($H^*$) is $w^*$. Since the Halmitonian cycle is constructed by adding an edge to the BST, the costliest edge in the BST cannot exceed $w^*$. Because if the extra edge added into the BST is no heavier than $w^*$, the BST has the same cost as the cycle. Otherwise if the extra edge is heavier than $w^*$, the cycle has more cost than the BST. \\
Thus, in G, we have $c_{max}(BST^*) \leq c_{max}(H^*) = w^*$. \\
And also note that a full walk (W) of the BST satisfies $c_{max}(W) \leq 3w^* = 3c_{max}(H^*)$. Because by definition of $G^3$, the costliest edge in the walk can correspondingly map to a path in G with at most 3 edges. Even for the worst case, all 3 edges are equally costliest, which is $w^*$. The costliest edge in the walk will not exceed $3w^*$. \\
Finally, apply the triangle inequality, skipping nodes in the walk will not increase the cost, which leads to
$c_{max}(H) \leq c_{max}(W)$. \\
To sum it up, there is $c_{max}(H) \leq c_{max}(W) \leq 3c_{max}(H^*)$. \\
From above, we see that this algorithm gives an approximation ratio of 3. \\

Now consider the time complexity, for a graph with $|V|$ vertices, there could exist at most $\frac{|V|*(|V|-1)}{2}$ edges. Thus, constructing $G^3$ can be done within polynomial time $O(|V|^2)$. \\
And from problem 23-3, we can compute the BST by removing edges starting from the costliest in linear time. Taking a full walk can also easily be done in polynomial time. So overall, this algorithm is polynomial.




\end{document}
